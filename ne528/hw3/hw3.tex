\documentclass{template}
\title{NE 795 Assignment 3}
\author{Kyle Hansen}
\date{8 October 2025}

\usepackage{cancel}

\begin{document}
\maketitle


\section{Plasma Based Ion Thrusters}

\subsection{Exit velocity}

A particle of charge $1\cdot q_e$ accelerated through a field with a potential difference of $500 \text{ V}$ will gain an energy of $500 \text{ eV}$. Then assuming $v_0 = 0$ and Xenon mass is $131.3\text{ amu}\cdot 931.5 \text{ MeV/c2 amu}$, the velocity gain at the end of the cylinder is

\begin{equation}
    \boxed{v_E = \sqrt{\frac{2E}{m}} = \sqrt{\frac{2\cdot 500 eV}{129.76 \text{ GeV/c2}}} = 8.18 \cdot 10^{-5}c = 2.711 \cdot 10^4 \text{m s}^{-1}.}
\end{equation}

The electric field strength can be found by dividing the potential difference by the length of the cylinder:

\begin{equation}
    \vec{E} = \frac{500\text{ V}}{0.15\text{ m}} = 3.333\cdot 10^3 \text{ V m}^{-1} = 3.333\cdot 10^3 \text{ N C}^{-1}
\end{equation}

Then the $E \times V$ velocity is:

\begin{equation}
    \boxed{|v_{E\times B}| = \frac{|\vec{E}_\perp\times\vec{B}|}{|B|^2} = \frac{2.5\times10^{-3}\cdot 3.33\times10^{3} }{\left(2.5\times10^{-3} \right)^2}\frac{\text{m}}{\text{s}} = 1.333\cdot 10^6 \frac{\text{m}}{\text{s}}
    }
\end{equation}

where the vectors are orthogonal, so $|\vec{E}\times\vec{B}| = |\vec{E}|\cdot|\vec{B}|$. The direction of the velocity, defining the $E$-field to be parallel with the $z$-axis, is clockwise motion (away from the page at the bottom of the cylinder, into the page at the top of the cylinder)-- tangent to the cylinder.

\subsection{Xe feed rate}

Since force can be expressed in a change in momentum over time $dp/dt$, the force a flow of Xe fuel can be calculated using the exit velocity:

\begin{equation}
    F = \frac{\Delta p}{\Delta t} = \frac{\Delta (m)}{\Delta t} = \Delta v \frac{\Delta m}{\Delta t} = v_{\text{exit}} \dot{m}
\end{equation}

then, accounting for ionization efficiency (def. $\eta$), the mass consumption rate is:

\begin{equation}
    \boxed{\dot{m} = \frac{F}{\eta \cdot v_{\text{exit}} } = \frac{0.08\text{ N}}{0.9 \cdot 2.711 \cdot 10^4 \text{ m s}^{-1}} = 3.280 \cdot 10^{-6} \; \frac{\text{kg}}{\text{s}}}
\end{equation}


\subsection{Chemical Rocket Equivalent}

Using the same formula as for the Xe consumption (where force is change in momentum), and assuming an efficiency of $\eta = 1$,

\begin{equation}
    \boxed{\frac{F}{\eta \cdot v_{\text{exit}} } = \frac{0.08\text{ N}}{9.0 \cdot 10^3 \text{ m s}^{-1}} = 8.89 \cdot 10^{-5} \; \frac{\text{kg}}{\text{s}}}
\end{equation}

or about 27 times the mass consumption rate for the ion thruster.

\section{Particle Trajectories}

Because there si no $\vec{I}$ field, the only variables to consider in this particle's equation of motion are Lorentz force $\vec{F} = q\left(\vec{v}\times\vec{B}\right)$ and the generalized drift velocity $\vec{v}_{drift} = \left(  \vec{v}_\parallel^2  + \frac{1}{2} \vec{v}_\perp^2 \right) \frac{\vec{B}\times\nabla\vec{B}}{\omega_c B^2}$. Since the magnetic field around a current is given by

\begin{equation}
    B = \frac{\mu_0 I}{2\pi r}
\end{equation}

and has a (right-handed) curved shape relative to the $+\vec{I}$-direction, some quantities can be found:

\begin{equation}
    \nabla\vec{B} = \frac{\partial \vec{B}}{\partial r} = -\frac{\mu_0 I}{2\pi}\frac{1}{r^2}
\end{equation}

(in the direction of $-r$, towards the wire)

\begin{equation}
    |\vec{B}\times\nabla\vec{B}| = \left(frac{\mu_0 I}{2\pi}\right)\frac{1}{r^3}.
\end{equation}

Then, the magnitude of $\vec{v}_{drift}$ is

\begin{equation}
    |\vec{v}_{drift}| = \left(  \vec{v}_\parallel^2  + \frac{1}{2} \vec{v}_\perp^2 \right)\frac{1}{\omega_c r},
\end{equation}

where $\vec{v}_\parallel$ is the velocity parallel to the $B$-field. Since $\vec{v}_0$ is parallel to the current, $\vec{v}_\parallel|_{t=0} = 0$ and $\vec{v}_\perp|_{t=0} = \vec{v}_0$.

The Lorentz force can be subdivided into its components, where $\vec{v} = \vec{v}_\parallel + \vec{v}_\perp$ (perpendicular and parallel to $\vec{B}$, implying no relation to $\vec{I}$). The force is then

\begin{equation}
    \frac{F} = q\left( \vec{v} \times \vec{B} \right) = \cancelto{0}{q\left( \vec{v}_\parallel \times \vec{B} \right)} + q\left( \vec{v}_\perp \times \vec{B} \right).
\end{equation}

Note that since the drift velocity is always parallel to $\vec{I}$ and the Lorentz force is perpendicular to the direction of the magnetic field, assuming no other force acts on the charge, the charge always stays in the same plane, and the position can be represented as only two variables, $r$ and $z$. Its velocity is also always perpendicular to $\vec{B}$.

\begin{equation}
    \vec{F} = q\left( \sqrt{v_r+v_z}\cdot B \right)\hat{n}
\end{equation}

where $\hat{n}$ is the unit vector in the direction of $\vec{v}\times\vec{B}$.



% \bibliographystyle{ieeetr}
% \bibliography{references.bib}





\end{document}