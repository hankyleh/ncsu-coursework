\documentclass{template}
\title{NE 795 Assignment 1}
\author{Kyle Hansen}
\date{26 September 2025}

\begin{document}
\maketitle

\section{Plasma Conditions}

\subsection{Conditions}

List the three conditions that describe the plasma state.

\subsection{}

Evaluate these conditions for the following systems given the electron density ($n_\e$),
electron temperature ($T_\e$) and plasma-neutral collision time ($\tau$):

\begin{itemize}
    \item The ionosphere ($n_\e = 1\cdot 10^{12}\text{ m}^{-3}, T_\e = 0.1\text{ eV}, \tau = 1\cdot 10^{-5}\text{ s}$)
    \item A candle flame ($n_\e = 1\cdot 10^{14}\text{ m}^{-3}, T_\e = 0.1\text{ eV}, \tau = 1\cdot 10^{-10}\text{ s}$)
\end{itemize}

Where
\[
\sqrt{\frac{\e^2}{\epsilon_0 m_\e}} = 56.41 \text{ m}^{3/2}\text{s}^-1
\]

\[
\sqrt{\frac{\epsilon_0}{\e^2}} = 7433.94\text{ eV}^{-1/2}\text{m}^{-1/2}
\]

with elementary charge ($e$) $1.602\cdot10^{-19}$ C, electron mass (me) $9.109\cdot10^{-31}$ kg and vacuum permittivity ($\epsilon_0$) $8.854 \cdot 10^{-12}$ Fm-1

\subsection{}

Conclude for both systems if they are a plasma (or not) and provide a reasoning for your conclusion.

\section{Debye Length}

Consider two infinite, parallel plates located at x = -d and x = d. Both plates are kept at a potential of $\Phi$ = 0V. The space between the plates is uniformly filled with a gas at density N of particles with charge q (note this is not a plasma, all the particles have the same charge).

\subsection{}

Using Poisson's equation, show that the potential distribution between the plates is given by the equation:

\[
    \Phi(x) = \frac{Nq}{2\epsilon_0}(d^2 - x^2)
\]


\subsection{}

The Debye length $\lambda_D$ describes the scale length at which the balance between electrostatic potential energy and thermal kinetic energy is established in a plasma. Show that for $d > \lambda_D$ the electrostatic potential energy needed to transport a particle from one of the plates to the midpoint at $x=0$ exceeds the average thermal kinetic energy of particles in a one-dimensional system. Assume a one-dimensional Maxwellian distribution with $<E_{kinetic}>=\frac{1}{2}k_B T$

\bibliographystyle{ieeetr}
\bibliography{references.bib}





\end{document}