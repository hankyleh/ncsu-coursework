\documentclass{template}
\title{NE 795 Assignment 2}
\author{Kyle Hansen}
\date{1 October 2025}

\usepackage{cancel}
\newcommand*\multbox[1]{\fbox{\hspace{0ex}#1\hspace{0ex}}}






\begin{document}

\maketitle

\section{} Show that 

\[
    I_\nu = ch\nu\psi_\nu.
\]

Intensity $I$ is defined as the power (rate of energy delivery) by radiation per unit solid angle, equal to:

\begin{equation}
        \text{Intensity} = \frac{\text{Energy}}{\text{particle}}\times\text{particle speed}\times\text{angular density}
\end{equation}

Where, for photons, speed is always $c$ and energy $E=h\nu$. Then

\begin{equation}
        \text{Intensity} = I_\nu = h\nu \times c\times\text{angular density}
\end{equation}

and the density is $\psi_\nu = \text{particles}\cdot\text{m}^{-3}\cdot \text{Sr}^{-1}$, and thus $I_\nu = ch\nu\psi_\nu$.

% TODO -- check units


\section{}

Show that the equilibrium intensity is equal to

\[
I_\nu = B_\nu = \frac{2h\nu^3}{c^2}\frac{1}{e^{h\nu/kT}-1}.
\]

% TODO


\section{} Derive the form of the following moments of the specific intensity:

\begin{gather*}
    E_\nu = \frac{1}{c} \int_{4\pi} I_\nu d\Omega, \qquad E = \int_0^\infty E_{\nu} d \nu,\\
    F_\nu = \int_{4\pi} \Omega I_\nu d\Omega,\\
    \mathbb{P}_\nu
\end{gather*}

\hrule

Given the Radiative Transfer equation:

\begin{equation}
    \frac{1}{c}\frac{\partial I_\nu}{\partial t} + \vec{\Omega}\cdot\vec{\nabla}I_\nu + \varkappa_\nu I\nu = \eta_\nu
\end{equation}

integrating over angle:

\begin{equation}
    \int_{4\pi} {\left\{\frac{1}{c}\frac{\partial I_\nu}{\partial t} + \vec{\Omega}\cdot\vec{\nabla}I_\nu + \varkappa_\nu I\nu = \eta_\nu\right\}} d\Omega
\end{equation}

becomes

\begin{equation}
    \frac{\partial E_\nu}{\partial t} + \int{\vec{\nabla}\cdot\left(\vec{\Omega} I_\nu\right)} d\Omega + c\varkappa_\nu E\nu =  4\pi \eta_\nu
\end{equation}

\begin{equation}
    \boxed{\frac{\partial E_\nu}{\partial t} + \vec{\nabla}F_\nu + c\varkappa_\nu E\nu = 4\pi \eta_\nu}
\end{equation}

(this is the monochromatic continuity equation)

The equation for $E = \int E_\nu d\nu$ is:

\begin{equation}
    \int_0^\infty \left\{\frac{\partial E_\nu}{\partial t} + \vec{\nabla}F_\nu + c\varkappa_\nu E\nu = 4\pi \eta_\nu\right\} d\nu
\end{equation}

\begin{equation}
    \boxed{\frac{\partial E}{\partial t} + \vec{\nabla}F + c  \int_0^\infty {\varkappa_\nu E\nu d\nu}=  \int_0^\infty{4\pi\eta_\nu d\nu}}
\end{equation}

(this is the energy-integrated continuity equation)

The equation for $F = \int \Omega I_\nu d\Omega $ is:

\begin{equation}
    \int_{4\pi} \Omega {\left\{\frac{1}{c}\frac{\partial I_\nu}{\partial t} + \vec{\Omega}\cdot\vec{\nabla}I_\nu + \varkappa_\nu I\nu = \eta_\nu\right\}} d\Omega
\end{equation}

\begin{equation}
    \frac{1}{c}\frac{\partial F_\nu}{\partial t} + \vec{\nabla}\int_{4\pi} {\Omega \otimes \Omega I_\nu d\Omega} + \varkappa_\nu F\nu = \cancelto{0}{\int_{4\pi} {\Omega \eta_\nu d\Omega}}
\end{equation}

where the source term reduces to zero (assuming isotropic source emission). The streaming term can be written in terms of the pressure tensor:

\begin{equation}
    \boxed{\frac{1}{c}\frac{\partial F_\nu}{\partial t} + c\vec{\nabla}\mathbb{P}_\nu + \varkappa_\nu F\nu = 0}
\end{equation}


% TODO -- field is in thermal equilibrium-- emission equals capture, redefine all previous progress.

\section{} Derive the speed of radiation wave in vacuum in the radiative transfer (RT) model defined by

\begin{itemize}
    \item The grey time-dependent $P_1$ equations
    \item The grey time-dependent $P_{1/3}$ equations
\end{itemize}

% TODO

\section{} Derive the system of the time-dependent $P_1$ and MEB equations in multigroup form from the spectral $P_1$ and MEB equations given by

\begin{gather*}
    \frac{\partial E_\nu}{\partial t} + \nabla F_\nu = c\varkappa_\nu E\nu = 4\pi \varkappa_\nu B_\nu\\
    \frac{\partial F_\nu}{\partial t} + \frac{1}{3}\nabla E_\nu + \varkappa_\nu F_\nu = 0\\
    \frac{\partial \varepsilon(T)}{\partial t} = \varkappa _ \nu (cE_\nu - 4\pi B_\nu)
\end{gather*}


Note that as part of this derivation you will need to define group opacities in the multigroup photon balance, first moment, and MEB equations.

% TODO

\end{document}
