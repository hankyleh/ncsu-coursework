\documentclass{template}
\title{NE 795 Assignment 2}
\author{Kyle Hansen}
\date{1 October 2025}

\usepackage{cancel}
\newcommand*\multbox[1]{\fbox{\hspace{0ex}#1\hspace{0ex}}}






\begin{document}

\maketitle

\section{} Show that 

\[
    I_\nu = ch\nu\psi_\nu.
\]

Intensity $I$ is defined as the power (rate of energy delivery) by radiation per unit solid angle, equal to:

\begin{equation}
        \text{Intensity} = \frac{\text{Energy}}{\text{particle}}\times\text{particle speed}\times\text{angular density}
\end{equation}

Where, for photons, speed is always $c$ and energy $E=h\nu$. Then

\begin{equation}
        \text{Intensity} = I_\nu = h\nu \times c\times\text{angular density}
\end{equation}

and the density is $\psi_\nu = \text{particles}\cdot\text{m}^{-3}\cdot \text{Sr}^{-1}$, and thus $I_\nu = ch\nu\psi_\nu$.

% TODO -- check units


\section{}

Show that the equilibrium intensity is equal to

\[
I_\nu = B_\nu = \frac{2h\nu^3}{c^2}\frac{1}{e^{h\nu/kT}-1}.
\]

Given the radiative transfer equation:

\begin{equation}
    \frac{1}{c}\frac{\partial I_\nu}{\partial t} + \vec{\Omega}\cdot\vec{\nabla}I_\nu + \varkappa_\nu I\nu = \eta_\nu,
\end{equation}

Where the function $\eta_\nu$ is the emission spectrum. Planck's and Kirchoff's laws, which describe the physics of photon emission, hold that the intensity of emitted radiation is

\begin{equation}
    \varkappa_\nu B_\nu = \varkappa_\nu B(\nu, T) = (\varkappa_\nu) \frac{2h\nu^3}{c^2}\frac{1}{e^{h\nu/kT}-1}.
\end{equation}

At equilibrium, the time and spatial derivatives both go to zero, so

\begin{equation}
    \varkappa_\nu B_\nu = \varkappa_\nu I_\nu
\end{equation}

or

\begin{equation}
    \boxed{B_\nu (\vec{r}, \vec{\Omega}) = I_\nu(\vec{r}, \vec{\Omega})}
\end{equation}


\section{} Derive the form of the following moments of the specific intensity:

\begin{gather*}
    E_\nu = \frac{1}{c} \int_{4\pi} I_\nu d\Omega, \qquad E = \int_0^\infty E_{\nu} d \nu,\\
    F_\nu = \int_{4\pi} \Omega I_\nu d\Omega,\\
    \mathbb{P}_\nu = \frac{1}{c} \int_{4\pi} \Omega \otimes \Omega I_\nu d\Omega, \qquad \mathbb{P} = \int_{0}^{\infty}\mathbb{P}_\nu d\nu
\end{gather*}

\hrule

At equilibrium, since $I_\nu = B_\nu$, the radiative transfer equation can be reduced to

\begin{equation}
    \frac{1}{c}\frac{\partial I_\nu}{\partial t} + \vec{\Omega}\cdot\vec{\nabla}I_\nu  = \varkappa_\nu B\nu- \varkappa_\nu I\nu = 0.
\end{equation}


Integrating over angle:

\begin{equation}
    \int_{4\pi} {\left\{\frac{1}{c}\frac{\partial I_\nu}{\partial t} + \vec{\Omega}\cdot\vec{\nabla}I_\nu =0\right\}} d\Omega
\end{equation}

becomes

\begin{equation}
    \frac{\partial E_\nu}{\partial t} + \int{\vec{\nabla}\cdot\left(\vec{\Omega} I_\nu\right)} d\Omega = 0
\end{equation}

\begin{equation}
    \boxed{\frac{\partial E_\nu}{\partial t} + \vec{\nabla}F_\nu = 0}
\end{equation}


(this is the monochromatic equilibrium continuity equation)

The equation for $E = \int E_\nu d\nu$ is:

\begin{equation}
    \int_0^\infty \left\{\frac{\partial E_\nu}{\partial t} + \vec{\nabla}F_\nu = 0 \right\} d\nu
\end{equation}

\begin{equation}
    \boxed{\frac{\partial E}{\partial t} + \vec{\nabla}F + c  = 0}
\end{equation}

(this is the energy-integrated equilibrium continuity equation).



The equation for $F = \int \Omega I_\nu d\Omega $ is:

\begin{equation}
    \int_{4\pi} \Omega {\left\{\frac{1}{c}\frac{\partial I_\nu}{\partial t} + \vec{\Omega}\cdot\vec{\nabla}I_\nu =0\right\}} d\Omega
\end{equation}

\begin{equation}
    \frac{1}{c}\frac{\partial F_\nu}{\partial t} + \vec{\nabla}\int_{4\pi} {\Omega \otimes \Omega I_\nu d\Omega} = 0
\end{equation}

\begin{equation}
    \boxed{\frac{1}{c}\frac{\partial F_\nu}{\partial t} + c\vec{\nabla}\mathbb{P}_\nu = 0}
\end{equation}

and the same equation integrated over frequency is
\begin{equation}
    \boxed{\frac{1}{c}\frac{\partial F}{\partial t} + c\vec{\nabla}\mathbb{P} = 0}
\end{equation}

Then the pressure tensor can be found similarly:

\begin{equation}
    \int_{4\pi} \Omega \otimes \Omega {\left\{\frac{1}{c}\frac{\partial I_\nu}{\partial t} + \vec{\Omega}\cdot\vec{\nabla}I_\nu =0\right\}} d\Omega
\end{equation}

\begin{equation}
    \boxed{\frac{\partial \mathbb{P_\nu}}{\partial t} +  \int_{4\pi} \Omega \otimes \Omega (\vec{\Omega}\cdot\vec{\nabla}I_\nu)d\Omega = 0}
\end{equation}

and the frequency-integrated equation:

\begin{equation}
    \boxed{\frac{\partial \mathbb{P}}{\partial t} +  \int_{4\pi} \Omega \otimes \Omega (\vec{\Omega}\cdot\vec{\nabla}I)d\Omega = 0}
\end{equation}

In the case of the $P_1$ equations, the pressure equation is closed by an approximation.




\section{} Derive the speed of radiation wave in vacuum in the radiative transfer (RT) model defined by

\begin{itemize}
    \item The gray time-dependent $P_1$ equations
    \item The gray time-dependent $P_{1/3}$ equations
\end{itemize}

In a vacuum, the gray $P_1$ equations are:

\begin{gather}\label{eq:gray-p1}
    \frac{\partial E}{\partial t} + \vec{\nabla} F = 0\\
    \frac{1}{c}\frac{\partial F}{\partial t} + \frac{c}{3}\vec{\nabla}E = 0
\end{gather}

and the gray $P_{1/3}$ equations are:

\begin{gather}\label{eq:gray-p13}
    \frac{\partial E}{\partial t} + \vec{\nabla} F = 0\\
    \frac{1}{3c}\frac{\partial F}{\partial t} + \frac{c}{3}\vec{\nabla}E = 0
\end{gather}


\subsection{} The grey time-dependent $P_1$ equations:

The speed of the wave can be found using the standard wave equation:

\begin{equation}\label{eq:wave-eqn}
    \frac{\partial^2 u}{\partial t^2} = v^2 \nabla^2 u
\end{equation}

For some quantity $u$ traveling at speed $v$. In this case, $u=E$. Then, from \autoref{eq:gray-p1}, the second time derivative is

\begin{equation}\label{eq:second-time-d}
    \frac{\partial^2 E}{\partial t^2} = -\nabla \frac{\partial F}{\partial t}
\end{equation}

where $\nabla F$ can be found from the second line of \autoref{eq:gray-p1}:

\begin{equation}
    \frac{1}{c}\frac{\partial F}{\partial t} = - \frac{c}{3}\vec{\nabla}E
\end{equation}
\begin{equation}
     \frac{\partial F}{\partial t} = - \frac{c^2}{3}\vec{\nabla}E
\end{equation}
\begin{equation}
    \nabla \frac{\partial F}{\partial t} = - \nabla \frac{c^2}{3}\nabla E = - \frac{c^2}{3}\nabla^2 E
\end{equation}

plugging back into \autoref{eq:second-time-d},

\begin{equation}
    \frac{\partial^2 E}{\partial t^2} = -\nabla \frac{c^2}{3}\nabla^2 E
\end{equation}

which takes the form of the wave equation (\autoref{eq:wave-eqn}), where $v^2 = c^2/3$ or

\begin{equation}
    \boxed{v = \frac{c}{\sqrt{3}}}
\end{equation}

\subsection{} The grey time-dependent $P_{1/3}$ equations:

Since the first equation of the two approximations are the same, the second time derivative of $E$ still isotropic

\begin{equation}\label{eq:p13-second-time-d}
    \frac{\partial^2 E}{\partial t^2} = -\nabla \frac{\partial F}{\partial t}
\end{equation}

the $\nabla F$ term can be found using the second line of \autoref{eq:gray-p13}:

\begin{equation}
    \frac{1}{3c}\frac{\partial F}{\partial t} = - \frac{c}{3}\vec{\nabla}E
\end{equation}
\begin{equation}
    \frac{\partial F}{\partial t} = - c^2\nabla E
\end{equation}
\begin{equation}
    \nabla \frac{\partial F}{\partial t} = - c^2\nabla^2 E
\end{equation}

Then, plugging into \autoref{eq:p13-second-time-d}, 

\begin{equation}
    \frac{\partial^2 E}{\partial t^2} = c^2\nabla^2 E
\end{equation}

This equation again takes the form of the wave equation, where

\begin{equation}
    \boxed{v = c}
\end{equation}




\section{} Derive the system of the time-dependent $P_1$ and MEB equations in multigroup form from the spectral $P_1$ and MEB equations given by

\begin{gather*}
    \frac{\partial E_\nu}{\partial t} + \nabla F_\nu + c\varkappa_\nu E\nu = 4\pi \varkappa_\nu B_\nu\\
    \frac{\partial F_\nu}{\partial t} + \frac{1}{3}\nabla E_\nu + \varkappa_\nu F_\nu = 0\\
    \frac{\partial \varepsilon(T)}{\partial t} = \varkappa_\nu (cE_\nu - 4\pi B_\nu)
\end{gather*}

Using the multigroup photon frequency convention:

\begin{gather*}
    I_p = \int_{\nu_p}^{\nu_{p+1}}I_\nu d\nu\\
    E_p = \int_{\nu_p}^{\nu_{p+1}}E_\nu d\nu\\
    F_p = \int_{\nu_p}^{\nu_{p+1}}F_\nu d\nu
\end{gather*}

with the frequency range groups 

\begin{gather*}
    \omega_p = [\nu_p, \nu_{p+1}]; \; p=1,2,\dots,N_p\\
    \nu_1 = 0; \; \nu_{N_p + 1} = \infty.
\end{gather*}

The zeroth moment is equal exactly to:

\begin{equation}
    \frac{\partial E_p}{\partial t} + \nabla F_p + c \int_p d\nu \; \varkappa_\nu E\nu = 4\pi \int_p d\nu \;\varkappa_\nu B_\nu.
\end{equation}

with the shorthand notation $\int_p = \int_{\nu_p}^{\nu_{p+1}}$. Substituting the opacities for their group-averaged opacities (weighted with the appropriate distribution) leaves the exact equation intact:

\begin{equation}
    \boxed{\frac{\partial E_p}{\partial t} + \nabla F_p + c \varkappa_p^E E_p = 4\pi \varkappa_p^B B_p}
\end{equation}

where the weighted opacities are:

\begin{align}
    \varkappa_p^B  &= \frac{\int_p d\nu \; \varkappa_\nu B_\nu}{\int_p d\nu \; B_\nu}\\
    \varkappa_p^E &=  \frac{\int_p d\nu \; \varkappa_\nu E_\nu}{\int_p d\nu \; E_\nu}
\end{align}


Integrating the first moment over group frequency, and using the group opacity approximation defined above, is:

\begin{equation}
    \boxed{\frac{\partial F_p}{\partial t} + \frac{1}{3}\nabla E_p + \varkappa_p^F F_\nu = 0}
\end{equation}

with the weighted opacity:

\begin{equation}
    \varkappa_p^F  = \frac{\int_p d\nu \; \varkappa_\nu F_\nu}{\int_p d\nu \; F_\nu}.
\end{equation}

Integrating the MEB equation and using the group opacities from before yields:

\begin{equation}
    \boxed{\frac{\partial \varepsilon(T)}{\partial t} =  c\varkappa_p^E E_\nu - 4\pi \varkappa_p^B B_\nu}
\end{equation}

This leaves the group opacities:

\begin{align}
    \varkappa_p^B  &= \frac{\int_p d\nu \; \varkappa_\nu B_\nu}{\int_p d\nu \; B_\nu}\\
    \varkappa_p^E &=  \frac{\int_p d\nu \; \varkappa_\nu E_\nu}{\int_p d\nu \; E_\nu}\\
    \varkappa_p^F  &= \frac{\int_p d\nu \; \varkappa_\nu F_\nu}{\int_p d\nu \; F_\nu}.
\end{align}


Since $B_\nu(T)$ is a known distribution, the group opacity $\varkappa_p^B$ can be evaluated exactly, but the other two cannot, and must be approximated. Near equilibrium, when $E_\nu \approx B_\nu$, then the same $\varkappa_p^B$ can be used-- this method offers the advantage that group opacities are set once, and do not need to be calcualted iteratively. Another approximation would be averaging this opacity over $B_\nu(T_{rad})$ using the equivalent temperature of the radiation at this point in phase space, rather than the physical temperature, Where

\begin{equation}
    T_{rad} = \left(\frac{1}{a_R}\int_0 ^\infty E_\nu\right)^{1/4} = \left(\frac{1}{a_R}\sum_{p=1}^{N_p} E_p\right)^{1/4}.
\end{equation}

Using this approximation, 

\begin{equation}
     \varkappa_p^E  \approx \frac{\int_p d\nu \; \varkappa_\nu B_\nu (T_{rad})}{\int_p d\nu \; B_\nu (T_{rad})}.
\end{equation}


\end{document}
