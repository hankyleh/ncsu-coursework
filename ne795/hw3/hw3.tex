\documentclass{template}
\title{NE 795 Assignment 3}
\author{Kyle Hansen}
\date{10 October 2025}

\usepackage{cancel}
\newcommand*\multbox[1]{\fbox{\hspace{0ex}#1\hspace{0ex}}}






\begin{document}

\maketitle

\section{Equilibrium Diffusion Limit}

The asymptotic analysis of the full TRT problem (in intensity rather than energy density) finds that:

\begin{equation*}
  (C_v + 4a_R T^3)\frac{\partial T}{\partial t} - \nabla \frac{4a_R T^3}{3\varkappa}\nabla T = 0
\end{equation*}

To order of $\epsilon^1$.

The $P_{1/3}$ equations are given by:

\begin{gather*}
  \frac{\partial E}{\partial t} + \vec{\nabla}\cdot\vec{F} + c \varkappa E = \varkappa a c T^4\\
  \frac{1}{3c}\frac{\partial F}{\partial t} + \frac{c}{3}\nabla E + \varkappa \vec{F} = 0\\
  c_v \frac{\partial T}{\partial t} = \varkappa c \left( E - aT^4 \right)
\end{gather*}

where $a = a_R$. In the asymptotic limit, where $\frac{\partial}{\partial t} \rightarrow \epsilon^2 \frac{\partial}{\partial t}$ and $\nabla \rightarrow \epsilon \nabla$, assuming the true distributions can be expanded as a power series in $\epsilon$, the expanded equations become:

\begin{gather}
  \epsilon^2 \sum \epsilon^n \frac{\partial E^{(n)}}{\partial t} + \epsilon\vec{\nabla} \cdot \sum \epsilon^n \vec{F^{(n)}} + c \varkappa \sum \epsilon^n E^{(n)} = \varkappa a c \sum \epsilon^n \left(T^4\right)^{(n)}\\
  \frac{\epsilon^2}{3c} \sum \epsilon^n\frac{\partial F^{(n)}}{\partial t} + \frac{\epsilon}{3} \sum \epsilon^n\nabla E^{(n)} + \varkappa \sum \epsilon^n \vec{F^{(n)}} = 0\\
  \epsilon^2 c_v \sum \epsilon^n \frac{\partial T^{(n)}}{\partial t} = \varkappa c \sum \epsilon^n \left( E^{(n)} - a\left(T^4\right)^{(n)} \right)
\end{gather}

Then, the $O(\epsilon^0)$ equations are:

\begin{equation}\label{eq:o0-energy}
  c\varkappa E^{(1)} = c\varkappa a \left(T^4 \right)^{(1)}
\end{equation}

\begin{equation}\label{eq:o0-flux}
   \varkappa \vec{F^{(1)}} = 0
\end{equation}

\begin{equation}\label{eq:o0-material}
  0 = \varkappa c \left[ E^{(0)} - a \left( T^4 \right)^{(0)} \right] 
\end{equation}


And the $O(\epsilon^1)$ equations are:

\begin{equation}\label{eq:o1-energy}
   \vec{\nabla}\vec{F}^{(0)} + c\varkappa E^{(1)} = c\varkappa a \left(T^4 \right)^{(1)}
\end{equation}

\begin{equation}\label{eq:o1-flux}
   \frac{c}{3}\nabla E^{(0)} + \varkappa \vec{F^{(1)}} = 0
\end{equation}

\begin{equation}\label{eq:o1-material}
  0 = \varkappa c \left[ E^{(1)} - a \left( T^4 \right)^{(1)} \right] 
\end{equation}

And the $O(\epsilon^2)$ equations are:

\begin{equation}\label{eq:o2-energy}
   \frac{\partial E^{(0)}}{\partial t} + \vec{\nabla}\vec{F}^{(1)} + c\varkappa E^{(2)} = c\varkappa a \left(T^4 \right)^{(2)}
\end{equation}

\begin{equation}\label{eq:o2-flux}
  \frac{1}{3c}\frac{\partial F^{(0)}}{\partial t} + \frac{c}{3}\nabla E^{(1)} + \varkappa \vec{F^{(2)}} = 0
\end{equation}

\begin{equation}\label{eq:o2-material}
  c_v \frac{\partial T^{(0)}}{\partial t} = \varkappa c \left[ E^{(2)} - a \left( T^4 \right)^{(2)} \right] 
\end{equation}

And the $O(\epsilon^3)$ equations are:

\begin{equation}\label{eq:o3-energy}
   \frac{\partial E^{(1)}}{\partial t} + \vec{\nabla}\vec{F}^{(2)} + c\varkappa E^{(3)} = c\varkappa a \left(T^4 \right)^{(3)}
\end{equation}

\begin{equation}\label{eq:o3-flux}
  \frac{1}{3c}\frac{\partial F^{(1)}}{\partial t} + \frac{c}{3}\nabla E^{(2)} + \varkappa \vec{F^{(3)}} = 0
\end{equation}

\begin{equation}\label{eq:o3-material}
  c_v \frac{\partial T^{(1)}}{\partial t} = \varkappa c \left[ E^{(3)} - a \left( T^4 \right)^{(3)} \right] 
\end{equation}


From these equations, some results are immediately apparant. From \autoref{eq:o0-energy} and \autoref{eq:o0-flux}, 

\begin{equation}
  \boxed{ E^{(0)} = a (T^4)^{(0)} = a\left(T^{(0)} \right)^4}
\end{equation}

\begin{equation}
  \boxed{   \vec{F}^{(0)} = 0.}
\end{equation}


From \autoref{eq:o1-material}, 

\begin{gather}
  0 = \varkappa c \left[ E^{(1)} - a \left( T^4 \right)^{(1)} \right]\\
  0 = \left[ E^{(1)} - a \left( T^4 \right)^{(1)} \right]\\
  \boxed{
    E^{(1)}  =     a \left( T^4 \right)^{(1)}  = 4a \left( T^{(0)} \right)^3 T^{(1)}}
\end{gather}

and from \autoref{eq:o1-flux}:

\begin{gather}
  \vec{F^{(1)}} = -\frac{c}{3 \varkappa} \nabla E^{(0)}\\
  \boxed{\vec{F^{(1)}} = -\frac{c a}{3 \varkappa} \nabla \left(T^{(0)} \right)^4}
\end{gather}

These results give the $O(\epsilon)$ and $O(1)$ asymptotic limits of energy density and flux. The temperature approximation can be found using \autoref{eq:o2-energy} and plugging in the results found above,


\begin{gather}
     \frac{\partial E^{(0)}}{\partial t} + \vec{\nabla}\vec{F}^{(1)} + c\varkappa E^{(2)} = c\varkappa a \left(T^4 \right)^{(2)}\\
  a \frac{\partial \left( T^{(0)} \right)^4}{\partial t} + \nabla \left( -\frac{c a}{3 \varkappa} \nabla {\left(T^{(0)} \right)}^4\right) + \cancelto{0}{ c \varkappa E^{(0)}} = \cancelto{0}{c \varkappa E^{(0)}} - c_v \frac{\partial T^{(0)}}{\partial t}\\
  c_v \frac{\partial T^{(0)}}{\partial t} + a \frac{\partial \left( T^{(0)} \right)^4}{\partial t} - \nabla\frac{ca}{3\varkappa}\nabla\left(T^{(0)} \right)^4 = 0
\end{gather}

and, since $\frac{\partial T^4}{\partial t} = 4T^3\frac{\partial T}{\partial t}$ and $\nabla T^4 = 4T^3 \nabla T$, this result reduces to 

\begin{equation} 
  \boxed{ \left( c_v + 4a\left(T^{(0)} \right)^3 \right) \frac{\partial T^{(0)}}{\partial t}  - \nabla \frac{4ca \left(T^{(0)} \right)^3}{3\varkappa} \nabla T^{(0)}. }
\end{equation}


Then, similarly evaluating \autoref{eq:o3-energy}, using terms from \autoref{eq:o1-material} and \autoref{eq:o2-flux}, and expanding $T^4$ terms as before,


\begin{gather}
  \frac{\partial E^{(1)}}{\partial t} + \vec{\nabla}\vec{F}^{(2)} + c\varkappa E^{(3)} = c\varkappa a \left(T^4 \right)^{(3)}\\
  \frac{\partial }{\partial t} \left( a \left( T^4 \right)^{(1)}  = 4a \left( T^{(0)} \right)^3 T^{(1)}\right) + \vec{\nabla} \left(  \cancelto{0}{-\frac{1}{3\varkappa c} \frac{\partial F^{(0)}}{\partial t} }    - \frac{c}{3\varkappa} \nabla E^{(1)}  \right) + \cancelto{0}{c\varkappa E^{(3)}} = \cancelto{0}{c\varkappa E^{(3)}} - c_v \frac{\partial T^{(1)}}{\partial t}\\
  \boxed{c_v \frac{\partial T^{(1)}}{\partial t} + 4a {T^{(0)}}^3 \frac{\partial  T^{(1)}}{\partial t} - \nabla \frac{4ca \left(T^{(0)} \right)^3}{3\varkappa} \nabla T^{(1)}}  
\end{gather}

Then, the $O(\epsilon)$ approximation of temperature is $T = T^{(0)} + \epsilon T^{(1)}$. Then, multiplying the $T^{(1)}$ equation by $\epsilon$ and adding to the $T^{(0)}$ equation yields the equilibrium diffusion limit which is consistent with the tranpsort limit:

\begin{equation}
  \boxed{ \left( c_v + 4aT^3 \right) \frac{\partial T}{\partial t}  - \nabla \frac{4ca T^3}{3\varkappa} \nabla T. }
\end{equation}



\clearpage
\section{Spectral Planck Function}


\begin{figure}[h]
  \centering
  \begin{subfigure}{0.48\textwidth}
        \includegraphics[width=\linewidth]{planck1.0.png}
        \caption{1 eV}
    \end{subfigure}
    \begin{subfigure}{0.48\textwidth}
        \includegraphics[width=\linewidth]{planck100.0.png}
        \caption{100 eV}
    \end{subfigure}

    \begin{subfigure}{0.48\textwidth}
        \includegraphics[width=\linewidth]{planck1000.0.png}
        \caption{1000 eV}
    \end{subfigure}
  \caption{Planck function}
\end{figure}



\clearpage
\section{Group Planck Function}

\begin{figure}[h]
  \centering
    \begin{subfigure}{0.48\textwidth}
        \includegraphics[width=\linewidth]{group_planck1.0.png}
        \caption{1 eV}
    \end{subfigure}
    \begin{subfigure}{0.48\textwidth}
        \includegraphics[width=\linewidth]{group_planck100.0.png}
        \caption{100 eV}
    \end{subfigure}

    \begin{subfigure}{0.48\textwidth}
        \includegraphics[width=\linewidth]{group_planck1000.0.png}
        \caption{1000 eV}
    \end{subfigure}
  \caption{Grouped planck function}
\end{figure}

\clearpage
\section{Group Planck Opacities}

\begin{figure}[h]
  \centering
  \begin{subfigure}{0.48\textwidth}
        \includegraphics[width=\linewidth]{planckopacities1.0.png}
        \caption{1 eV}
    \end{subfigure}
  \begin{subfigure}{0.48\textwidth}
        \includegraphics[width=\linewidth]{planckopacities100.0.png}
        \caption{100 eV}
    \end{subfigure}

  \begin{subfigure}{0.48\textwidth}
        \includegraphics[width=\linewidth]{planckopacities1000.0.png}
        \caption{1000 eV}
    \end{subfigure}
  \caption{Group Planck Opacities}
\end{figure}



\clearpage
\section{Group Rosseland Opacities}

Algebraic manipulation yields a formula for Rosseland opacities when using Fleck-Cummings opacity:

\begin{equation}
  \varkappa_{g,R} =  \frac{\varkappa^\star}{h^3}   \frac{ \int_g d\nu \;   \nu^4 {\left(  1 - e^{-h\nu/kT}  \right)}^{-2}   }{     \int_g d\nu \;   \nu^7 {\left(  1 - e^{-h\nu/kT}  \right)}^{-3}  }
\end{equation}

which I numerically integrated using \verb|scipy.integrate.quad()|.

\begin{figure}[h]
  \centering
  
  \begin{subfigure}{0.48\textwidth}
        \includegraphics[width=\linewidth]{rosseland1.0.png}
        \caption{1 eV}
  \end{subfigure}
  \begin{subfigure}{0.48\textwidth}
        \includegraphics[width=\linewidth]{rosseland100.0.png}
        \caption{100 eV}
  \end{subfigure}

  \begin{subfigure}{0.48\textwidth}
        \includegraphics[width=\linewidth]{rosseland1000.0.png}
        \caption{1000 eV}
  \end{subfigure}
  \caption{Group Rosseland Opacities}
\end{figure}

\begin{figure}[h]
  \centering

  \begin{subfigure}{0.48\textwidth}
        \includegraphics[width=\linewidth]{comparison1.0.png}
        \caption{1 eV}
  \end{subfigure}
  \begin{subfigure}{0.48\textwidth}
        \includegraphics[width=\linewidth]{comparison100.0.png}
        \caption{100 eV}
  \end{subfigure}

  \begin{subfigure}{0.48\textwidth}
        \includegraphics[width=\linewidth]{comparison1000.0.png}
        \caption{1000 eV}
  \end{subfigure}
  \caption{Grouped Opacities}
\end{figure}

\end{document}
